\chapter*{Resumo}
\addcontentsline{toc}{chapter}{\protect\numberline{}Resumo}

\newthought{No presente trabalho}, exploramos dois temas de interesse em finanças, economia e antropologia social, através da aplicação de técnicas da teoria da informação e da mecânica estatística.

\newthought{No primeiro tópico}, estudamos a conexão entre teoria de dependência estatística, teoria de informação e teoria da cópulas. O conceito de distribuição cópula é revisto e aplicado em reformulação das definições de medida de dependência dadas por Rényi \cite{Renyi1959}. Em seguida, mostramos que a informação mútua satisfaz todos os requisitos para ser uma boa medida de dependência. Obtemos uma identidade entre a informação mútua e a entropia da distribuição cópula e uma decomposição mais específica da informação mútua de uma distribuição elíptica nas suas partes linear e não-linear. Avaliamos o risco de usar quantidades ingênuas como medidas de dependência estatística, mostrando que a correlação linear pode subestimar grosseiramente a dependência. Esses resultados são utilizados para desenvolver um método de detectação de desvios de dependência gaussiana em pares de variáveis ​​aleatórias e aplicá-lo a séries temporais financeiras. Finalmente, discutimos um método para ajustar t-cópulas a dados empíricos \cite{Calsaverini2009} através da medida da informação mútua e do tau de Kendall.

\newthought{No segundo tópico}, desenvolvemos um modelo para o surgimento de autoridade em sociedades humanas pré-agrícolas. Discutimos motivações empíricas com raízes em neurociência, primatologia e antropologia para um modelo matemático capaz de explicar a ampla variabilidade de formas de organização social humana no eixo igualitário-hierárquico. O modelo resulta da aplicação de teoria da informação a uma hipótese sobre os custos evolutivos envolvidos na vida social. O modelo gera um diagrama de fase rico, com diferentes regimes que podem ser interpretados como diferentes tipos de organização social. Os parâmetros de controler do modelo estão ligados à capacidade cognitiva da espécie em questão, ao tamanho do grupo e a pressões ecológicas e sociais.

\chapter*{Abstract}
\addcontentsline{toc}{chapter}{\protect\numberline{}Abstract}
 
\newthought{In the present work} we explore two topics of interest in finance, economics and social anthropology through the application of techniques from information theory and statistical mechanics. 

\newthought{In the first topic} we study the connexion between statistical dependency theory, information theory and copula theory. The concept of copula distribution is reviewed and applied to the reformulation of the definition of dependency measures given by Rényi \cite{Renyi1959}. It is then shown that the mutual information satisfy all the requirements to be a good dependency measure. We derive an identity between mutual information and the entropy of the copula distribution and a more specific decomposition of the mutual information of an elliptical distribution into its linear and non-linear parts. We evaluate the risk of using naive measures as statistical dependency measures by showing that linear correlation can grossly underestimate dependency. Those results are used to develop a method to detect deviation from gaussian dependence in pairs of random variables and apply it to financial time series. Finally, we discuss a method to adjust t-copulas to empirical data\cite{Calsaverini2009} through the determination of the mutual information and Kendall's tau.

\newthought{In the second topic} we develop a model for the emergence of authority in pre-agricultural human societies. We discuss empirical motivations with roots in neuroscience, primatology and anthropology for a mathematical model able to explain the ample variability of forms of human social organization in the egalitarian-hierarchical axis. The model results from the application of information theory on a hypothesis about the evolutive costs involved in social life. It generates a rich phase diagram, with different regimes which can be interpreted as different types of societal organization, from egalitarian to hierarchical. The control parameters of the model are connected to the cognitive capacity of the species in question, the size of the group and ecological and social pressures.