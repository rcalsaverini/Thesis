\chapter*{Resumo}
\addcontentsline{toc}{chapter}{\protect\numberline{}Resumo}
Neste trabalho exploramos dois tópicos de aplicação de teoria de informação e mecanica estatística a problemas de interesse em finanças, economia e sociologia. No primeiro tópico exploramos a conexão entre a teoria de dependência estatística e a teoria de informação através da teoria de cópulas. Após uma revisão do conceito de cópulas, reformulamos a definição de medidas de dependência de Renyi\cite{Renyi1959} usando esse conceito e mostramos que a informação mútua satisfaz todos os requisitos para ser uma medida de dependência. Em seguida mostramos uma relação entre a informação mútua e a entropia da distribuição cópula, e uma relação mais específica para a decomposição da informação mútua de distribuições elípticas em uma parte devido à dependência gaussiana linear e uma parte não-linear. As conseqüências dessas duas decomposições sobre o risco do uso de pseudo-medidas de dependências são então discutidas. Esses resultados são usados para desenvolver um método para detectar desvio de gaussianidade na dependência de séries temporais e para ajuste de cópulas t sobre dados empiricos\cite{Calsaverini2009}.

No segundo tópico desenvolvemos um modelo para emergência de autoridade em sociedades humanas. Discutimos as motivações empíricas com raízes na neurociência, na primatologia e na antropologia para um modelo matemático que explique o espectro amplo de tipos de organização social humana no eixo igualitário-hierárquico. O modelo resulta da aplicação de teoria de informação sobre uma hipótese sobre os custos evolutivos envolvidos. O modelo apresenta um diagrama de fases rico, com diferentes regimes que podem ser interpretadas como correspondendo a diferentes tipos de organização social, desde igualitária até hierárquica. Os parâmetros de controle do sistema são identificados com a capacidade cognitiva da espécie em questão e as pressões ecológica e social em que o grupo está imerso.

\chapter*{Abstract}
\addcontentsline{toc}{chapter}{\protect\numberline{}Abstract}
 
In this work we explore two topics of interest in the application of information theory and statistical mechanics techniques to problems in finance, economics and sociology. In the fisrt topic we study the conexion between statistical dependency theory and information theory mediated by copula theory. After a revision of the concept of copulas, we reformulate the definition of dependency measures given by Renyi \cite{Renyi1959} using this concepyt and show that mutual information satisfy all the requirements to be a dependency measure. We then show a relationship between mutual information and copula entropy, and a more specific decomposition of the mutual information of an elliptical distribution into its linear and non-linear parts. We evaluate the risk of using naive pseudo-dependency measures. Those results are then used to develop a method to detect deviation from gaussianity in the dependency of time series and a method to adjust t-copulas to data\cite{Calsaverini2009}.

On the second topic we develop a model for the emergence of authority in early human societies. We discuss empirical motivations with roots in neuroscience, primatology and anthropology for a mathematical model able to explain the spectrum of observed types of human social organization in the egalitarian-hierarchical axis. The model results from the application of information theory on a hypothesys about the evolutive costs involved in social life. The model generates a rich phase diagram, with diferent regimes which can be interpreted as different types of societal organization, from egalitarian to hierarchical. The control parameters of the model are connected to the cognitive capacity of the species in question and ecological 	and social pressures.