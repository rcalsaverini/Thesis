\chapter{Teoremas de Cox}
\label{ap:provateoremas}

Este apêndice contém demonstrações dos teoremas exibidos na seção \ref{sec:probabilidadeseinferencia}, \emph{\nameref{sec:probabilidadeseinferencia}}. O método de demonstração consiste em considerar casos em que há informação completa, ou seja, em que é válida a lógica tradicional, em que existam duas formas diferentes de determinar a mesma plausibilidade. Impondo consistência sobre essas situações, se obtém equações funcionais cujas soluções implicam no resultado desejado. 

\section{Primeiro teorema de Cox e a regra do produto} 
\begin{Teorema}
    Uma vez que uma representação consistente de plausibilidades $(P|Q)$ com um ordenamento bem definido foi encontrada, sempre é possível encontrar uma outra equivalente $\pi(P|Q)$, de forma que:
    \begin{equation}
	\label{eq:productrule2}
	\pi(P_1\wedge P_2|Q) = \pi(P2 | Q \wedge P_1) \pi(P_1| Q)
    \end{equation}
\begin{proof}
A função $G(u,v)$ define a decomposição da plausibilidade de uma conjunção de duas proposições:
\[
 (P_1\wedge P_2|Q) = G((P_1|Q), (P_2|Q\wedge P_2)).
\]
A conjunção booleana ($\wedge$) é uma operação associativa, ou seja:
\[
  P_1\wedge (P_2 \wedge P_3) =  (P_1\wedge P_2) \wedge P_3.
\] 
Isso implica também na associatividade da função $G(u,v)$, ou seja:
\[
  G(G(u,v),w) = G(u,G(v,w)). 
\]
Esse vínculo é satisfeito por infinitas possíveis funções $G(u,v)$, porém todas elas \cite{Aczel1975, ACaticha2008} têm a forma:
\[
  G(u,v)  = g^{-1}(g(u) g(v)) 
\]
com $g(\cdot)$ uma função monotônica. Sendo $g(u)$ monotônica, pode-se redefinir a atribuição de números reais às plausibilidades para $g(P|Q)$ sem perder o ordenamento de proposições segundo suas plausibilidades. Escrevendo portanto:
\[
 \pi(P|Q) = g((P|Q))
\]
se obtém a equação \eqref{eq:productrule2}, que é reminiscente da regra do produto da Teoria das Probabilidades:
\[
  \pi(P_1\wedge P_2|Q) = \pi(P2 | Q \wedge P_1) \pi(P_1| Q)
\]
\end{proof}
\end{Teorema}

\section{Valores extremos} 
\begin{Teorema}
 Sejam\footnote{Das palavras inglesas \emph{``true''} e \emph{``false''}, respectivamente.} $P_T$ e $P_F$ os valores associados à plausibilidade regraduada $\pi(\cdot| \cdot)$ de eventos sabidamente verdadeiros ou falsos, respectivamente\footnote{Requisitos de consistência exigem que sejam iguais para quaisquer proposições falsas ou verdadeiras}. Temos necessariamente que:
 \begin{align}
  P_T &= 1 \text{ e } P_F = 0  \text{, ou}\\
  P_T &= 1 \text{ e } P_F = \infty
 \end{align}


\begin{proof}
Se $P$ é sabido verdadeiro, então a plausibilidade de que $P$ e $Q$ sejam simultaneamente verdadeiros é exatamente a plausibilidade de apenas $Q$ ser verdadeiro, ou seja, $\pi(P\wedge Q|P) = \pi(Q|P)$. Mas, pela regra do produto:
\[
 \pi(P\wedge Q|P) = \pi(P|P) \pi(Q|P\wedge P) = P_T \pi(Q|P) 
\]
Dessa forma, $P_T \pi(Q|P) = \pi(Q|P)$, para quaisquer $Q$ e $P$, o que implica em $P_T = 1$. 

Da mesma forma, a plausibilidade de que simultaneamente $P$ e $\bar{P}$ sejam verdadeiros, dada uma proposição $Q$ qualquer, deve ser $P_F$, pois $P\wedge\bar{P}$ é uma contradição. Mas, pela regra do produto:
\[
 \pi(P\wedge\bar{P}|Q) = \pi(P | Q\wedge\bar{P}) \pi(\bar{P}|Q)
\]
Independentemente de $Q$, $\pi(P|Q\wedge\bar{P})$ deve ser também igual a $P_F$ e, assim, $P_F = P_F \pi(\bar{P}|Q)$, para quaisquer $P$ e $Q$. Duas soluções são possíveis: $P_F = 0$ ou $P_F = \infty$. Uma vez que quaisquer das soluções para $P_F$ pode ser mapeada na outra por uma regraduação monotônica\footnote{Consistindo da função $h(u) = \frac{1}{u}$.}, pode-se arbitrariamente escolher $P_F = 0$, e assim limitar valores de $\pi(P|Q)$ no intervalo $[0,1]$. 
\end{proof}
\end{Teorema}

\section{Teorema de Bayes}
\begin{Teorema}
Para qualquer conjunto de três proposições $P_1$, $P_2$ e $Q$, temos:
\begin{equation}
 \label{eq:regradebayes2}
 \pi(P2 | Q \wedge P_1)  = \frac{\pi(P1 | Q \wedge P_2) \pi(P_2| Q)}{\pi(P_1| Q)}.
\end{equation}
\begin{proof}

Uma conseqüência imediata da regra do produto segue da seguinte observação. Uma vez que a conjunção $P\wedge Q$ é simétrica:
\[
 \pi(P_1\wedge P_2| Q) = \pi(P_2\wedge P_1 | Q) 
\]
Aplicando a regra do produto em ambos os membros da equação acima, temos:
\[
 \pi(P2 | Q \wedge P_1) \pi(P_1| Q) = \pi(P1 | Q \wedge P_2) \pi(P_2| Q)
\]
que implica trivialmente na equação \eqref{eq:regradebayes2} que é similar ao teorema de Bayes da Teoria de Probabilidades.
\end{proof}
\end{Teorema}

\section{Regra da soma}

Para demonstrar a regra da soma são necessários os seguintes lemas:
\begin{Lema}
\label{lema:demorgan}
Dadas as proposições $P$, $S$ e $Q = \overline{P\wedge S}$ temos que:
\begin{align}
  P\wedge\bar{Q}  & = \bar{Q} \\
  \bar{P}\wedge Q &= \bar{P}
\end{align}

\begin{proof}
As equações decorrem diretamente das leis de De Morgan\footnote{Leis de De Morgan são as seguintes relações distributivas da negação: \[\overline{A\wedge B} =\bar{A}\vee\bar{B},\] e, similarmente, \[\overline{A\vee B} = \bar{A}\wedge\bar{B},\] válidas para quaisquer proposições $A$ e $B$.}. Note, em primeiro lugar, que: 
\[
  P\wedge\bar{Q} = P \wedge (P \wedge S) = P \wedge S = \bar{Q}.
\]
Note ainda que:
\[
 \overline{\bar{P}\wedge Q} = P \vee \bar{Q} = P \vee (P \wedge S) = P,
\]
e portanto $\bar{P}\wedge Q = \bar{P}$ e $P\wedge\bar{Q} = \bar{Q}$.
\end{proof}
\end{Lema}

\begin{Teorema}[2º teorema de regraduação de Cox]
 Uma vez que uma representação consistente de plausibilidades $\pi(P|Q)$ com um ordenamento bem definido foi encontrada para a qual vale a regra do produto, sempre é possível encontrar uma outra equivalente $p(P|Q)$ tal que:
 \begin{equation}
 p(P|Q) + p(\bar{P} | Q) = 1 
 \end{equation}
\begin{proof}
A função $F(u,v)$ define a relação entre a plausibilidade de uma proposição e a plausibilidade de sua negação:
\[
  (\bar{P}|Q) = F(({P}|Q)).
\]
Considere a seguinte plausibilidade regraduada dada por \[\pi(P \wedge Q | R) = \pi(P|R) \pi(Q|P \wedge R).\] Note que a função $F(\cdot)$ deve ser idempotente, uma vez que $\bar{\bar{P}} = P$ implica que $F(F(u)) = u$. Portanto, 
\[
\pi(Q|S) = F(\pi(\bar{Q}|S)) 
\]
para qualquer $S$. Assim, usando repetidamente a regra do produto:
\begin{align}
\pi(P\wedge Q | R) &= \pi(P|R) F\left(\pi(\bar{Q}|P\wedge R)\right) \\
&= \pi(P|R) F\left(\frac{\pi(\bar{Q}\wedge P|R)}{\pi(P|R)}\right).
\end{align}
Mas a mesma operação pode ser feita em outra ordem --- uma vez que a conjunção $P\wedge Q$ é simétrica na troca de $P$ por $Q$ e, portanto:
\[
\pi(P\wedge Q | R) = \pi(P|R) F\left(\frac{\pi(\bar{Q}\wedge P|R)}{\pi(P|R)}\right) = \pi(Q|R) F\left(\frac{\pi(\bar{P}\wedge Q|R)}{\pi(Q|R)}\right)
\]
para quaisquer $P$, $Q$ e $R$. Em particular, deve valer para o caso particular em que $Q = \overline{P\wedge S}$. Nesse caso se aplicam os resultados do Lema \ref{lema:demorgan} e, portanto,  $P\wedge\bar{Q} = \bar{Q}$ e $\bar{P}\wedge Q = \bar{P}$. Isso permite escrever:
\[
\pi(P|R) F\left(\frac{\pi(\bar{Q}|R)}{\pi(P|R)}\right) = \pi(Q|R) F\left(\frac{\pi(\bar{P}|R)}{\pi(Q|R)}\right)
\]
e, finalmente, sendo arbitrárias as proposições $P$ e $S$, isso significa que:
\[
u F\left(\frac{F(v)}{u}\right) = v F\left(\frac{F(u)}{v}\right)
\]
Novamente, há infinitas soluções $F(\cdot)$ para esse vínculo, mas todas elas satisfazem\cite{Aczel1975, ACaticha2008}:
\begin{equation}
 F(u)^\alpha + u^\alpha = 1, 
\end{equation}
para algum $\alpha \in \mathbb{R}$. É possível, portanto, regraduar as plausibilidades novamente por uma transformação monotônica $p(P|Q) = \pi(P|Q)^\alpha$. Se $\alpha > 0$, essa regraduação preserva os valores $P_F = 0$ e $P_V = 1$ e o teorema de Bayes. Caso $\alpha<0$, sempre é possível aplicar a regraduação $h(u) = \frac{1}{u}$, que também é monotônica, e preservar os valores limite. Esse resultado permite enunciar o segundo teorema de Cox:
\begin{equation*}
  p(P|Q) + p(\bar{P} | Q) = 1 
\end{equation*}
\end{proof}
\end{Teorema}
