\chapter{Teorema de Sklar}

\newthought{Apresentamos nesse apêndice}, inicialmente, uma ``mostração'' informal e intuitiva do teorema de Sklar para duas variáveis e, logo depois, uma demonstração mais rigorosa. Em primeiro lugar, o teorema é enunciado da seguinte forma:

\begin{Teorema}[Teorema de Sklar.]
Sejam $X$ e $Y$ duas variáveis aleatórias com distribuição cumulativa conjunta $F(X,Y)$ e distribuições marginais $F_x(X)$ e $F_y(Y)$ respectivamente. Então:
\begin{itemize}
 \item Existe uma única função $C : [0,1]^2 \to \mathbb{R}$, chamada função cópula, tal que:
 \begin{equation}
  F(X, Y)  = C(F_x(X), F_y(Y)).
 \end{equation}
 Ou, de forma recíproca:
 \begin{equation}
 C(u,v) = F(F^{-1}_x(u), F^{-1}_y(v))
 \end{equation}
 
 A função $C$, denominada cópula, é uma distribuição cumulativa com suporte em $[0,1]$.
 
 \item De forma recíproca, se $C(u,v)$ é uma função cópula, ou seja, uma distribuição cumulativa com suporte em $[0,1]$, e se $F_x(X)$ e $F_y(Y)$ são distribuições cumulativas univariadas com suporte em $\mathcal{X}$ e $\mathcal{Y}$ respectivamente, então a função definida por $F(X, Y)  = C(F_x(X), F_y(Y))$ é uma distribuição com suporte em $\mathcal{X} \times \mathcal{Y}$ e distribuições marginais $F_x(X)$ e $F_y(Y)$.
\end{itemize}
\end{Teorema}

\section{Uma ``demonstração'' informal do teorema de Sklar}

Um ``demonstração'' informal, baseada em \citet{Ruschendorf2009}\cite{Ruschendorf2009}, auxilia na compreensão do conceito de cópula. Inicialmente, vamos definir a \emph{transformação distribucional}:
\begin{Definicao}[Transformação distribucional]

askdjsdj

\end{Definicao}



\newpage
O status de distribuição faz com que as seguintes identidades sobre $C(u,v)$ possam ser derivadas:
 \begin{itemize}
 \item $C(0,v) = C(u,0) = 0$,
 \item $C(1,v) = v$ e $C(u, 1) = u$
 \item $C(u,v)$ é uma função 2-crescente, ou seja: \[C(y_1,y_2)-C(x_1,y_2)-C(y_1,x_2)+C(x_1,x_2) \geq 0\] para todo $[x_1,y_1]\times[x_2,y_2]\subseteq [0,1]\times[0,1]$.
 \end{itemize}