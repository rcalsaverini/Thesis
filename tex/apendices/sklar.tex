\chapter{Teorema de Sklar}

\newthought{Apresentamos neste apêndice} uma demonstração do teorema de Sklar baseada nas referências \citep{Ruschendorf2009}\cite{Ruschendorf2009} e \citep{Faugeras2012}\cite{Faugeras2012}. Em primeiro lugar, o teorema é enunciado da seguinte forma:

\begin{Teorema}[Teorema de Sklar.]
Sejam $X$ e $Y$ duas variáveis aleatórias com distribuição cumulativa conjunta $F(x,y) = \Prob{X < x , Y < y}$ e distribuições marginais $F_x(X)$ e $F_y(Y)$ respectivamente. Então:
\begin{enumerate}
 \item Existe uma única função $C : [0,1]^2 \to \mathbb{R}$, chamada função cópula, tal que:
 \begin{equation}
 \label{eq:copulaproof}
  F(x, y)  = C(F_X(x), F_y(y)).
 \end{equation}
 Ou, de forma recíproca:
 \begin{equation}
 C(u,v) = F(F^{-1}_x(u), F^{-1}_y(v))
 \end{equation}
 
 A função $C$, denominada cópula, é uma distribuição cumulativa com suporte em $[0,1]$.
 
 \item De forma recíproca, se $C(u,v)$ é uma função cópula, ou seja, uma distribuição cumulativa com suporte em $[0,1]$, e se $F_X(x)$ e $F_Y(y)$ são distribuições cumulativas univariadas com suporte em $\mathcal{X}$ e $\mathcal{Y}$ respectivamente, então a função definida por $F(x, y)  = C(F_X(x), F_Y(y))$ é uma distribuição com suporte em $\mathcal{X} \times \mathcal{Y}$ e distribuições marginais $F_X(x)$ e $F_Y(y)$.
\end{enumerate}

\begin{proof}[Demonstração: caso contínuo]
Inicialmente note que, caso $F(x,y)$ seja contínua, \eqref{eq:copulaproof} pode ser demonstrada de forma simples com a transformação de variáveis:
\[
 U = F_X(X) \text{ e } V = F_Y(Y). 
\]
Note que $U$ e $V$ tomam valores no intervalo $[0,1]$ e são uniformemente distribuídas. Para provar o teorema de Sklar no caso contínuo, note que:
\begin{align}
 \label{eq:continous}
  F(x, y) = & \Prob{X < x , Y < y}  \nonumber \\ 
          = & \Prob{F_X(X) < F_X(x), F_Y(Y) < F_Y(y)}
\end{align}

uma vez que as funções distribuição $F_X(x)$ e $F_Y(y)$ são monotônicas e contínuas. Seja $C(u,v) = \Prob{U < u, V < v}$ a distribuição conjunta de $U$ e $V$. Dessa forma:
\[
 F(x, y) = \Prob{U < F_X(x), V < F_Y(y)} = C(F_X(x), F_Y(y)),
\]
demonstrando assim a primeira parte do teorema. A segunda parte é trivialmente demonstrada notando que os seguintes fatos: 
\begin{itemize}
 \item as distribuições marginais de $C(u,v)$ são uniformes no intervalo $[0,1]$, o que implica que $F(x,y) = C(F_X(x), F_Y(y))$ é uma distribuição cumulativa, uma vez que $0\le F_X(x)\le1$ e $0\le F_Y(y)\le1$ são elas próprias distribuições cumulativas 
 \item como $C(u, v)$ é uma distribuição cumulativa, então $C(1, v) = v$ e $C(u,1)= u$, o que implica que as marginais de $F(x,y)$ são dadas por
 \[
  F_X(x) = \Prob{X < x} = F(x, \infty) = C(F_X(x), 1) = F_X(x)
 \]
 e similarmente para $F_Y(y)$.
\end{itemize}
\end{proof}
\begin{proof}[Demonstração: caso descontínuo]
O ponto onde a demonstração anterior falha para o caso de descontinuidades\footnote{Por exemplo, para variáveis aleatórias que tomam valores em conjuntos discretos, entre outros casos.} em $F(x,y)$  é na equação \eqref{eq:continous}. Para contornar esse problema, vamos definir a seguinte transformação de variáveis:

\begin{Definicao}[Transformação distribucional] 
Seja $X$ uma variável aleatória com distribuição cumulativa $F(x) = \Prob{ X \le x}$, com suporte em $\mathcal{X}$. Seja a função $f(x, \lambda)$ dada por:
\[
 f(x, \lambda) = \Prob{X \le x} + \lambda \Prob{X = x}.
\]
Define-se a transformação distribucional de X como a variável aleatória:
\[
 U = f(X, Z) 
\]
onde $Z$ é uma variável aleatória uniformemente distribuida no intervalo $[0,1]$.
\end{Definicao}

Para o caso em que $F(x)$ é contínua, $f(x, u) = F(x)$ e $U = F(X)$ é idêntica à variável aleatória homônica definida na demonstração anterior. Para outros casos, essa identidade não se observa, porém ainda se sustentam as seguintes propriedades\footnote{Uma demonstração desse fato pode ser encontrada em \citet{Ruschendorf2009}}:
\begin{itemize}
 \item $U = F(X, z)$ é uniformemente distribuída no intervalo $[0,1]$. 
 \item $X = F^{-1}(U)$, onde $F^{-1}(\cdot)$ é a função quantil, definida por:
 \[
  F^{-1}(u) = \inf\{x \in \mathbb{R} | F(x) \ge u\}
 \]
\end{itemize}
De posse dessa definição podemos prosseguir com uma demonstração mais robusta do teorema de Sklar. Sejam $X$ e $Y$ variáveis aleatórias com função distribuição conjunta dada por $F(x, y)$. Sejam $Z$ uma variável uniformemente distribuídas no intervalo $[0,1]$. Finalmente, sejam $U = f_X(X,Z)$ e $V = f_Y(Y,Z)$ as transformações distribucionais associadas a $X$ e $Y$. Com anteriormente dito, são válidas as identidades $X = F_X^{-1}(U)$ e $Y = F_Y^{-1}(V)$. Dessa forma, denotando por $C(u,v)$ a distribuição conjunta de $U$ e $V$, temos:
\begin{align*}
 F(x,y) & = \Prob{X < x, Y < y} \\ &= \Prob{F_X^{-1}(U) < x , F_Y^{-1}(V)} \\&= \Prob{U < F_X(x), V < F_Y(y)}\\
        & = C(F_X(x), F_Y(y))
\end{align*}
A demonstração da segunda parte do teorema segue similarmente ao caso contínuo. 
\end{proof}
\end{Teorema}


